\section{Introduction}
\label{sec:intro}
In recent decades, quantum computing has made significant strides \cite{preskill2023quantum}, leveraging quantum mechanics to potentially revolutionize fields like cryptography, optimization, and machine learning. Despite these advances, current Noisy Intermediate Scale Quantum (NISQ) devices have yet to achieve quantum advantage, where quantum computers outperform classical ones in solving specific problems efficiently.

Currently, the most promising approaches for achieving meaningful quantum advantage are found in "Hybrid Quantum-Classical Computing" (HQCC), which merges the computational power of quantum devices with the reliability of classical systems. Variational Quantum Algorithms (VQAs) have emerged as a leading area of research within this framework, primarily due to their compatibility with near-term quantum systems. They demand fewer qubits and operate with shallower circuit depths, making them highly adaptable to current quantum technology. These algorithms are intentionally designed as hybrids, leveraging classical optimizers to fine-tune the parameters of Parameterized Quantum Circuits (PQCs).

Among the applications, combinatorial optimization has seen notable progress. The Quantum Approximate Optimization Algorithm (QAOA) \cite{farhi2014quantum} shows potential for solving the MaxCut problem, but requires more qubits than current devices can handle when scaled to larger problem instances. This limitation has prompted the development of alternative approaches such as the Qubit-Efficient MaxCut Heuristic Algorithm (QEMC) \cite{tenecohen2023variational}, which addresses MaxCut with fewer qubits. Continued research in hybrid quantum-classical methods is essential for achieving quantum advantage.

\subsection{Motivation}
\label{section:motivation}
% I might have to reduce the size of this section to fit the 10-page limit.
The search for efficient algorithms to solve combinatorial optimization problems is critical in numerous fields, such as logistics, finance, and telecommunications. The MaxCut problem, for example, has broad applications in areas like machine learning \cite{937505}, statistical physics \cite{Barahona_Grötschel_Jünger_Reinelt_1988}, circuit design \cite{Barahona_Grötschel_Jünger_Reinelt_1988}, and data clustering \cite{10.1007/11893318_21}. Creating more efficient algorithms to solve this problem can improve the performance of these applications, driving substantial progress in their respective fields.

Moreover, there is a strong interest from the computer science community in terms of computational complexity, particularly regarding the MaxCut problem's $NP$-hard nature. Efficient algorithms for this problem could shed light on the classical-to-quantum computing boundary and even the $P = NP$ problem. Imagining a world where $P = NP$ is proven true, though unlikely, is fascinating. It would imply polynomial-time solutions for all $NP$ problems, including MaxCut, Traveling Salesman Problem (TSP), and Knapsack Problem, among others. This breakthrough would revolutionize optimization, benefiting areas like vehicle routing and cryptography. Notably, RSA encryption could be compromised if $P = NP$, highlighting the critical importance of understanding these complexities.

The aforementioned considerations drive our efforts to develop a new algorithm for solving the MaxCut problem with greater efficiency and accuracy.

\subsection{Topic Overview}
\label{section:overview}
% I might have to reduce the size of this section to fit the 10-page limit.
In this project, we propose the Interpolated QAOA/QEMC Framework. This will enable us to develop novel VQAs by leveraging the strengths of two existing VQAs, QAOA and QEMC, for improved performance in solving the MaxCut problem. QAOA requires a qubit for each graph vertex, making it difficult to scale. In contrast, QEMC uses exponentially fewer qubits by assigning one basis state to each graph node, requiring only $\log_2(n)$ qubits (for $n$ graph vertices). However, this compression leads to limitations in QEMC's results. By interpolating both VQAs, we aim to create an algorithm that utilizes fewer qubits than QAOA and performs better than QAOA and QEMC. The new algorithm, constructed through the iQAQE Framework, assigns multiple basis states to each node, in contrast to QEMC's single basis state approach. This design tentatively allows for a more practical implementation on present-day NISQ devices, thanks to its reduced qubit requirements compared to QAOA, fewer measurement shots than QEMC, and potentially greater trainability than QAOA.

% \subsection{Objectives}
% \label{section:objectives}
% I might have to reduce the size of this section to fit the 10-page limit.
% The primary objective of this work is to develop and analyze the iQAQE Framework. Algorithms built within this framework will be implemented and tested using classical simulations of quantum machines. The deliverables for this project include the iQAQE Framework's code\footnote{Reach out to the author to obtain access to the code.}, the results obtained from the simulations, and the analysis of these results. The expected outcomes are improvements in the performance of the iQAQE algorithm compared to QAOA and QEMC, with a focus on accuracy, efficiency, and scalability.