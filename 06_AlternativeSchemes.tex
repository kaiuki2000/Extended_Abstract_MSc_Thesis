\section{Alternative Schemes}
\label{sec:Alternative_Schemes}

% Maybe I'll remove this section altogether, due to space limitations. I think I'll just briefly mention this somewhere else. Maybe at the end of the previous (iQAQE Schemes and Results) section.

We also developed alternative schemes that, although not entirely based on the iQAQE Framework, emerged from our research on these topics. These are different algorithms proposed to solve the MaxCut problem. Two such alternatives are the Parity-like QAOA and the Batch-based Oracle coloring scheme. The former involves "reducing" QAOA's problem Hamiltonian to use fewer qubits than usual, by using parity encodings, which allow us to compress the representation of \(N\) nodes into \(n < N\) qubits (with \(N = \mathcal{O}(n^k)\), for any \(k\), as long as \(\binom{n}{k} \geq N\)). The latter could be implemented with a trainable Oracle, which would receive $k < n$ qubits, and return the colors of $k$ nodes. Afterwards, we would run $\frac{n}{k}$ batches of this, to obtain all the $n$ nodes' colors. Due to space constraints, we won't discuss these schemes in detail in this extended abstract; we simply present them. For a detailed discussion, please see Chapter 6 of the thesis.