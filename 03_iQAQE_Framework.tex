\section{iQAQE Framework}
\label{sec: Methodology}

We introduce a new framework, inspired by QAOA and QEMC, for seamlessly designing multiple distinct VQAs. This framework builds on QEMC's core components while integrating concepts from QAOA to leverage the strengths of both algorithms for improved performance. Tentatively named the iQAQE Framework, it opens the door to numerous unexplored and unknown VQAs, offering a variety of parameters to experiment with, such as the number of qubits, list cardinality, and mappings.

In iQAQE, we depart from the QEMC approach by associating each graph node with a list of basis states, in contrast to QEMC's consideration of a single basis state for each node. Each of these lists comprises $c \in \left[1, 2^{n-1}\right]$ basis states, where $n$ represents the number of qubits. $c$ denotes the cardinality of the lists. What we formerly referred to as "mapping" involves distributing basis states among these lists. The term "mapping" can also signify a specific allocation of basis states among the lists. This design allows for potential overlap among states from different lists. Additionally, it is important to note that the encoding of these states will utilize a qubit range expected to fall between the QEMC and QAOA requirements, specifically in $[\log_2{N}, N]$, for an $N$-node graph.

As a default approach, we compute node probabilities straightforwardly by summing the probabilities of associated basis states and then normalizing the result. This is necessary because we utilize QEMC's probability threshold encoding scheme. Additionally, we consider iQAQE to use the same ansatz and cost function as QEMC, adjusted for the appropriate number of qubits. However, we also explore the potential for slight deviations from this formula, such as alternative methods for computing node probabilities and the adoption of problem-inspired ansatz variations.

We can postulate that such a hybrid approach might have some potential advantages. Namely, it should allow for less shots than in QEMC and fewer qubit requirements than in QAOA, in addition to, arguably, being better trainable \cite{tenecohen2023variational}. Another notable feature of this algorithm is its tunable "quantum-ness." By interpolating between QAOA (a hybrid quantum-classical method) and QEMC (a classical, quantum-inspired method), we can selectively adjust the degree of quantum and classical components, which could prove to be advantageous.

We designate this as a framework due to the vast array of possibilities in qubit number, list cardinality, and mappings, which facilitate the creation of multiple unique VQAs, each with its own merits and drawbacks. In this study, we will present heuristics for parameter selection and provide a comprehensive analysis of the corresponding results.

% Maybe, I should explain why we believe this should yield better results. Less qubits, shallower circuits, tunable "quantum-ness", etc.

% The primary objective of this study is to ascertain the optimal mapping from basis states to lists, essentially determining which basis states should be included in specific lists/nodes. Naturally, this undertaking also involves the identification of a suitable cost function and ansatz to ensure the algorithm's effective operation. For most of our work, we consider iQAQE to use the same ansatz and cost function as QEMC, adjusted for the appropriate number of qubits and with some modifications that we'll discuss below.