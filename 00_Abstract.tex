%%%%%%%%%%%%%%%%%%%%%%%%%%%%%%%%%%%%%%%%%%%%%%%%%%%%%%%%%%%%%%%%%%%%%%
%     File: ExtendedAbstract_abstr.tex                               %
%     Tex Master: ExtendedAbstract.tex                               %
%                                                                    %
%     Author: Andre Calado Marta                                     %
%     Last modified : 2 Dez 2011                                     %
%%%%%%%%%%%%%%%%%%%%%%%%%%%%%%%%%%%%%%%%%%%%%%%%%%%%%%%%%%%%%%%%%%%%%%
% The abstract of should have less than 500 words.
% The keywords should be typed here (three to five keywords).
%%%%%%%%%%%%%%%%%%%%%%%%%%%%%%%%%%%%%%%%%%%%%%%%%%%%%%%%%%%%%%%%%%%%%%

%%
%% Abstract
%%
\begin{abstract}
    % A little more concise:
    In this work, we introduce the Interpolated QAOA/QEMC (iQAQE) Framework, a novel approach inspired by the Quantum Approximate Optimization Algorithm (QAOA) and the Qubit-Efficient MaxCut Heuristic Algorithm (QEMC) for designing Variational Quantum Algorithms (VQAs) to solve the Maximum Cut (MaxCut) problem. This Framework builds on the core components of QEMC while integrating concepts from QAOA to harness the strengths of both algorithms, requiring fewer qubits and demonstrating greater resilience to statistical uncertainty associated with small shot numbers compared to QEMC. This framework provides adjustable parameters to create various VQAs, and we introduce heuristics for selecting these parameters, including the number of qubits, list cardinality, and mappings. Our evaluations show that iQAQE often matches or exceeds the performance of QEMC and can outperform classical algorithms like Goemans-Williamson in specific scenarios. We also propose two alternative approaches for solving the MaxCut problem derived from our iQAQE research, which offer additional insights and potential research directions. Additionally, we present a small machine learning model to determine optimal mappings for specific graphs based on statistical properties of the mappings themselves. Ultimately, the iQAQE Framework serves as a versatile testbed for developing new VQAs, potentially leading to significant advancements in quantum computing. \\

    % Thesis's abstract:
    % In this work, we introduce the Interpolated QAOA/QEMC (iQAQE) Framework, a novel approach inspired by the Quantum Approximate Optimization Algorithm (QAOA) and the Qubit-Efficient MaxCut Heuristic Algorithm (QEMC), for designing multiple distinct Variational Quantum Algorithms (VQAs) to solve the Maximum Cut (MaxCut) problem. This framework builds on the core components of QEMC while integrating concepts from QAOA to harness the strengths of both algorithms. The iQAQE Framework requires fewer qubits than QAOA and exhibits greater resilience to statistical uncertainty associated with small shot numbers compared to QEMC. The framework offers a range of adjustable parameters, facilitating the creation of various VQAs. We introduce heuristics for selecting these parameters, such as the number of qubits, list cardinality, and mappings, and evaluate their performance. Our findings indicate that iQAQE often performs on par with QEMC and can even surpass classical state-of-the-art algorithms like Goemans-Williamson in certain scenarios. Additionally, we propose two alternative approaches for solving the MaxCut problem, derived from our research on iQAQE. While these methods do not fall directly within the iQAQE Framework, they offer valuable insights and potential avenues for future research. We also present a small machine learning model designed to determine the optimal mapping for a specific graph based on statistical properties of the mappings themselves. Ultimately, the iQAQE Framework serves as a versatile testbed for developing new VQAs, potentially paving the way for groundbreaking results in the future. \\

%%
%% Keywords (max 5)
%%
\noindent{{\bf Keywords:}} Hybrid Quantum-Classical Computing (HQCC), Variational Quantum Algorithms (VQAs), Maximum Cut (MaxCut) Problem, Quantum Approximate Optimization Algorithm (QAOA), Qubit-Efficient MaxCut Heuristic Algorithm (QEMC). \\

\end{abstract}

