\section{Conclusions}
\label{sec:Conclusions}
% Results paragraph:
The primary accomplishment of this study is the development of the iQAQE Framework, offering promising competition with current state-of-the-art algorithms. We've showcased its advantages over QAOA and QEMC, particularly in terms of reduced qubit requirements and enhanced resilience to statistical uncertainty. Our introduced heuristics have generally shown competitive results, even surpassing classical state-of-the-art (GW) in certain scenarios, like ND-CNOT-based QEMC for a $32$-node graph. This success underscores the potential of the iQAQE Framework. Moreover, we proposed a method for determining optimal mappings based on statistical properties, utilizing a small machine learning model. While initial results were not entirely satisfactory, we believe this approach can inspire further research.

% Future work paragraph:
Future work could focus on enhancing the machine learning model, developing new heuristics, and testing on larger graphs. Exploring neural networks, alternative input variables, different heuristics configurations, and problem-inspired ansätze could yield valuable insights. Additionally, testing the framework on real-world quantum computers is essential for assessing its practical viability. Further refinement and testing of proposed heuristics will also be crucial for continued improvement. Another research direction is to continue developing the alternative schemes presented in section \ref{sec:Alternative_Schemes}, which could provide valuable insights and potential avenues for future study. By addressing these areas, future research can build upon the foundation laid by this work, potentially leading to significant advancements in VQAs.

% Additionally, testing the framework on real-world quantum computers is essential for assessing its practical viability.
% Think of this as: "The algorithm's performance could be very significantly affected by the hardware's noise, rendering it less effective than expected. Therefore, testing on real-world quantum computers is essential for assessing its practical viability."

\section*{Acknowledgments}
\label{sec:Acknowledgments}
This work was developed in collaboration with Zoltán Zimborás and Bence Bakó, from the Wigner Research Centre for Physics (Hungary), in the context of project: \textit{HQCC – Hybrid Quantum-Classical Computing}, supported by the EU QuantERA ERA-NET Co-fund in Quantum Technologies and by FCT -- Funda\c{c}\~{a}o para a Ci\^{e}ncia e a Tecnologia (QuantERA/004/2021). Additionally, I would like to thank my research supervisor, Yasser Omar, and my colleague, Miguel Murça, for their invaluable guidance in this project.

% Re-read everything and make sure it's all good. Might remove some parts in the introduction/background to make space for more results.

% Should also re-read my thesis one more time, to make sure I didn't mess anything up, when copying stuff from there to here.

\nocite{Tabi_2020,nielsen2010quantum,sciorilli2024largescale,Karp2010,NP-problems}